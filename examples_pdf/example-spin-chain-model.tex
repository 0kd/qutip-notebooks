
% Default to the notebook output style

    


% Inherit from the specified cell style.




    
\documentclass{article}

    
    
    \usepackage{graphicx} % Used to insert images
    \usepackage{adjustbox} % Used to constrain images to a maximum size 
    \usepackage{color} % Allow colors to be defined
    \usepackage{enumerate} % Needed for markdown enumerations to work
    \usepackage{geometry} % Used to adjust the document margins
    \usepackage{amsmath} % Equations
    \usepackage{amssymb} % Equations
    \usepackage[mathletters]{ucs} % Extended unicode (utf-8) support
    \usepackage[utf8x]{inputenc} % Allow utf-8 characters in the tex document
    \usepackage{fancyvrb} % verbatim replacement that allows latex
    \usepackage{grffile} % extends the file name processing of package graphics 
                         % to support a larger range 
    % The hyperref package gives us a pdf with properly built
    % internal navigation ('pdf bookmarks' for the table of contents,
    % internal cross-reference links, web links for URLs, etc.)
    \usepackage{hyperref}
    \usepackage{longtable} % longtable support required by pandoc >1.10
    

    
    
    \definecolor{orange}{cmyk}{0,0.4,0.8,0.2}
    \definecolor{darkorange}{rgb}{.71,0.21,0.01}
    \definecolor{darkgreen}{rgb}{.12,.54,.11}
    \definecolor{myteal}{rgb}{.26, .44, .56}
    \definecolor{gray}{gray}{0.45}
    \definecolor{lightgray}{gray}{.95}
    \definecolor{mediumgray}{gray}{.8}
    \definecolor{inputbackground}{rgb}{.95, .95, .85}
    \definecolor{outputbackground}{rgb}{.95, .95, .95}
    \definecolor{traceback}{rgb}{1, .95, .95}
    % ansi colors
    \definecolor{red}{rgb}{.6,0,0}
    \definecolor{green}{rgb}{0,.65,0}
    \definecolor{brown}{rgb}{0.6,0.6,0}
    \definecolor{blue}{rgb}{0,.145,.698}
    \definecolor{purple}{rgb}{.698,.145,.698}
    \definecolor{cyan}{rgb}{0,.698,.698}
    \definecolor{lightgray}{gray}{0.5}
    
    % bright ansi colors
    \definecolor{darkgray}{gray}{0.25}
    \definecolor{lightred}{rgb}{1.0,0.39,0.28}
    \definecolor{lightgreen}{rgb}{0.48,0.99,0.0}
    \definecolor{lightblue}{rgb}{0.53,0.81,0.92}
    \definecolor{lightpurple}{rgb}{0.87,0.63,0.87}
    \definecolor{lightcyan}{rgb}{0.5,1.0,0.83}
    
    % commands and environments needed by pandoc snippets
    % extracted from the output of `pandoc -s`
    \DefineVerbatimEnvironment{Highlighting}{Verbatim}{commandchars=\\\{\}}
    % Add ',fontsize=\small' for more characters per line
    \newenvironment{Shaded}{}{}
    \newcommand{\KeywordTok}[1]{\textcolor[rgb]{0.00,0.44,0.13}{\textbf{{#1}}}}
    \newcommand{\DataTypeTok}[1]{\textcolor[rgb]{0.56,0.13,0.00}{{#1}}}
    \newcommand{\DecValTok}[1]{\textcolor[rgb]{0.25,0.63,0.44}{{#1}}}
    \newcommand{\BaseNTok}[1]{\textcolor[rgb]{0.25,0.63,0.44}{{#1}}}
    \newcommand{\FloatTok}[1]{\textcolor[rgb]{0.25,0.63,0.44}{{#1}}}
    \newcommand{\CharTok}[1]{\textcolor[rgb]{0.25,0.44,0.63}{{#1}}}
    \newcommand{\StringTok}[1]{\textcolor[rgb]{0.25,0.44,0.63}{{#1}}}
    \newcommand{\CommentTok}[1]{\textcolor[rgb]{0.38,0.63,0.69}{\textit{{#1}}}}
    \newcommand{\OtherTok}[1]{\textcolor[rgb]{0.00,0.44,0.13}{{#1}}}
    \newcommand{\AlertTok}[1]{\textcolor[rgb]{1.00,0.00,0.00}{\textbf{{#1}}}}
    \newcommand{\FunctionTok}[1]{\textcolor[rgb]{0.02,0.16,0.49}{{#1}}}
    \newcommand{\RegionMarkerTok}[1]{{#1}}
    \newcommand{\ErrorTok}[1]{\textcolor[rgb]{1.00,0.00,0.00}{\textbf{{#1}}}}
    \newcommand{\NormalTok}[1]{{#1}}
    
    % Define a nice break command that doesn't care if a line doesn't already
    % exist.
    \def\br{\hspace*{\fill} \\* }
    % Math Jax compatability definitions
    \def\gt{>}
    \def\lt{<}
    % Document parameters
    \title{example-spin-chain-model}
    
    
    

    % Pygments definitions
    
\makeatletter
\def\PY@reset{\let\PY@it=\relax \let\PY@bf=\relax%
    \let\PY@ul=\relax \let\PY@tc=\relax%
    \let\PY@bc=\relax \let\PY@ff=\relax}
\def\PY@tok#1{\csname PY@tok@#1\endcsname}
\def\PY@toks#1+{\ifx\relax#1\empty\else%
    \PY@tok{#1}\expandafter\PY@toks\fi}
\def\PY@do#1{\PY@bc{\PY@tc{\PY@ul{%
    \PY@it{\PY@bf{\PY@ff{#1}}}}}}}
\def\PY#1#2{\PY@reset\PY@toks#1+\relax+\PY@do{#2}}

\expandafter\def\csname PY@tok@sh\endcsname{\def\PY@tc##1{\textcolor[rgb]{0.73,0.13,0.13}{##1}}}
\expandafter\def\csname PY@tok@nb\endcsname{\def\PY@tc##1{\textcolor[rgb]{0.00,0.50,0.00}{##1}}}
\expandafter\def\csname PY@tok@s\endcsname{\def\PY@tc##1{\textcolor[rgb]{0.73,0.13,0.13}{##1}}}
\expandafter\def\csname PY@tok@w\endcsname{\def\PY@tc##1{\textcolor[rgb]{0.73,0.73,0.73}{##1}}}
\expandafter\def\csname PY@tok@gs\endcsname{\let\PY@bf=\textbf}
\expandafter\def\csname PY@tok@si\endcsname{\let\PY@bf=\textbf\def\PY@tc##1{\textcolor[rgb]{0.73,0.40,0.53}{##1}}}
\expandafter\def\csname PY@tok@vc\endcsname{\def\PY@tc##1{\textcolor[rgb]{0.10,0.09,0.49}{##1}}}
\expandafter\def\csname PY@tok@gu\endcsname{\let\PY@bf=\textbf\def\PY@tc##1{\textcolor[rgb]{0.50,0.00,0.50}{##1}}}
\expandafter\def\csname PY@tok@mi\endcsname{\def\PY@tc##1{\textcolor[rgb]{0.40,0.40,0.40}{##1}}}
\expandafter\def\csname PY@tok@ge\endcsname{\let\PY@it=\textit}
\expandafter\def\csname PY@tok@sb\endcsname{\def\PY@tc##1{\textcolor[rgb]{0.73,0.13,0.13}{##1}}}
\expandafter\def\csname PY@tok@kt\endcsname{\def\PY@tc##1{\textcolor[rgb]{0.69,0.00,0.25}{##1}}}
\expandafter\def\csname PY@tok@gt\endcsname{\def\PY@tc##1{\textcolor[rgb]{0.00,0.27,0.87}{##1}}}
\expandafter\def\csname PY@tok@mh\endcsname{\def\PY@tc##1{\textcolor[rgb]{0.40,0.40,0.40}{##1}}}
\expandafter\def\csname PY@tok@nt\endcsname{\let\PY@bf=\textbf\def\PY@tc##1{\textcolor[rgb]{0.00,0.50,0.00}{##1}}}
\expandafter\def\csname PY@tok@o\endcsname{\def\PY@tc##1{\textcolor[rgb]{0.40,0.40,0.40}{##1}}}
\expandafter\def\csname PY@tok@cs\endcsname{\let\PY@it=\textit\def\PY@tc##1{\textcolor[rgb]{0.25,0.50,0.50}{##1}}}
\expandafter\def\csname PY@tok@ow\endcsname{\let\PY@bf=\textbf\def\PY@tc##1{\textcolor[rgb]{0.67,0.13,1.00}{##1}}}
\expandafter\def\csname PY@tok@nd\endcsname{\def\PY@tc##1{\textcolor[rgb]{0.67,0.13,1.00}{##1}}}
\expandafter\def\csname PY@tok@cp\endcsname{\def\PY@tc##1{\textcolor[rgb]{0.74,0.48,0.00}{##1}}}
\expandafter\def\csname PY@tok@kr\endcsname{\let\PY@bf=\textbf\def\PY@tc##1{\textcolor[rgb]{0.00,0.50,0.00}{##1}}}
\expandafter\def\csname PY@tok@bp\endcsname{\def\PY@tc##1{\textcolor[rgb]{0.00,0.50,0.00}{##1}}}
\expandafter\def\csname PY@tok@sx\endcsname{\def\PY@tc##1{\textcolor[rgb]{0.00,0.50,0.00}{##1}}}
\expandafter\def\csname PY@tok@c\endcsname{\let\PY@it=\textit\def\PY@tc##1{\textcolor[rgb]{0.25,0.50,0.50}{##1}}}
\expandafter\def\csname PY@tok@s2\endcsname{\def\PY@tc##1{\textcolor[rgb]{0.73,0.13,0.13}{##1}}}
\expandafter\def\csname PY@tok@mo\endcsname{\def\PY@tc##1{\textcolor[rgb]{0.40,0.40,0.40}{##1}}}
\expandafter\def\csname PY@tok@kp\endcsname{\def\PY@tc##1{\textcolor[rgb]{0.00,0.50,0.00}{##1}}}
\expandafter\def\csname PY@tok@na\endcsname{\def\PY@tc##1{\textcolor[rgb]{0.49,0.56,0.16}{##1}}}
\expandafter\def\csname PY@tok@kd\endcsname{\let\PY@bf=\textbf\def\PY@tc##1{\textcolor[rgb]{0.00,0.50,0.00}{##1}}}
\expandafter\def\csname PY@tok@mf\endcsname{\def\PY@tc##1{\textcolor[rgb]{0.40,0.40,0.40}{##1}}}
\expandafter\def\csname PY@tok@go\endcsname{\def\PY@tc##1{\textcolor[rgb]{0.53,0.53,0.53}{##1}}}
\expandafter\def\csname PY@tok@gd\endcsname{\def\PY@tc##1{\textcolor[rgb]{0.63,0.00,0.00}{##1}}}
\expandafter\def\csname PY@tok@nv\endcsname{\def\PY@tc##1{\textcolor[rgb]{0.10,0.09,0.49}{##1}}}
\expandafter\def\csname PY@tok@sc\endcsname{\def\PY@tc##1{\textcolor[rgb]{0.73,0.13,0.13}{##1}}}
\expandafter\def\csname PY@tok@sr\endcsname{\def\PY@tc##1{\textcolor[rgb]{0.73,0.40,0.53}{##1}}}
\expandafter\def\csname PY@tok@kn\endcsname{\let\PY@bf=\textbf\def\PY@tc##1{\textcolor[rgb]{0.00,0.50,0.00}{##1}}}
\expandafter\def\csname PY@tok@nf\endcsname{\def\PY@tc##1{\textcolor[rgb]{0.00,0.00,1.00}{##1}}}
\expandafter\def\csname PY@tok@gr\endcsname{\def\PY@tc##1{\textcolor[rgb]{1.00,0.00,0.00}{##1}}}
\expandafter\def\csname PY@tok@ss\endcsname{\def\PY@tc##1{\textcolor[rgb]{0.10,0.09,0.49}{##1}}}
\expandafter\def\csname PY@tok@cm\endcsname{\let\PY@it=\textit\def\PY@tc##1{\textcolor[rgb]{0.25,0.50,0.50}{##1}}}
\expandafter\def\csname PY@tok@gp\endcsname{\let\PY@bf=\textbf\def\PY@tc##1{\textcolor[rgb]{0.00,0.00,0.50}{##1}}}
\expandafter\def\csname PY@tok@k\endcsname{\let\PY@bf=\textbf\def\PY@tc##1{\textcolor[rgb]{0.00,0.50,0.00}{##1}}}
\expandafter\def\csname PY@tok@vg\endcsname{\def\PY@tc##1{\textcolor[rgb]{0.10,0.09,0.49}{##1}}}
\expandafter\def\csname PY@tok@ne\endcsname{\let\PY@bf=\textbf\def\PY@tc##1{\textcolor[rgb]{0.82,0.25,0.23}{##1}}}
\expandafter\def\csname PY@tok@m\endcsname{\def\PY@tc##1{\textcolor[rgb]{0.40,0.40,0.40}{##1}}}
\expandafter\def\csname PY@tok@err\endcsname{\def\PY@bc##1{\setlength{\fboxsep}{0pt}\fcolorbox[rgb]{1.00,0.00,0.00}{1,1,1}{\strut ##1}}}
\expandafter\def\csname PY@tok@s1\endcsname{\def\PY@tc##1{\textcolor[rgb]{0.73,0.13,0.13}{##1}}}
\expandafter\def\csname PY@tok@vi\endcsname{\def\PY@tc##1{\textcolor[rgb]{0.10,0.09,0.49}{##1}}}
\expandafter\def\csname PY@tok@sd\endcsname{\let\PY@it=\textit\def\PY@tc##1{\textcolor[rgb]{0.73,0.13,0.13}{##1}}}
\expandafter\def\csname PY@tok@nn\endcsname{\let\PY@bf=\textbf\def\PY@tc##1{\textcolor[rgb]{0.00,0.00,1.00}{##1}}}
\expandafter\def\csname PY@tok@nc\endcsname{\let\PY@bf=\textbf\def\PY@tc##1{\textcolor[rgb]{0.00,0.00,1.00}{##1}}}
\expandafter\def\csname PY@tok@il\endcsname{\def\PY@tc##1{\textcolor[rgb]{0.40,0.40,0.40}{##1}}}
\expandafter\def\csname PY@tok@nl\endcsname{\def\PY@tc##1{\textcolor[rgb]{0.63,0.63,0.00}{##1}}}
\expandafter\def\csname PY@tok@gh\endcsname{\let\PY@bf=\textbf\def\PY@tc##1{\textcolor[rgb]{0.00,0.00,0.50}{##1}}}
\expandafter\def\csname PY@tok@kc\endcsname{\let\PY@bf=\textbf\def\PY@tc##1{\textcolor[rgb]{0.00,0.50,0.00}{##1}}}
\expandafter\def\csname PY@tok@ni\endcsname{\let\PY@bf=\textbf\def\PY@tc##1{\textcolor[rgb]{0.60,0.60,0.60}{##1}}}
\expandafter\def\csname PY@tok@no\endcsname{\def\PY@tc##1{\textcolor[rgb]{0.53,0.00,0.00}{##1}}}
\expandafter\def\csname PY@tok@gi\endcsname{\def\PY@tc##1{\textcolor[rgb]{0.00,0.63,0.00}{##1}}}
\expandafter\def\csname PY@tok@c1\endcsname{\let\PY@it=\textit\def\PY@tc##1{\textcolor[rgb]{0.25,0.50,0.50}{##1}}}
\expandafter\def\csname PY@tok@se\endcsname{\let\PY@bf=\textbf\def\PY@tc##1{\textcolor[rgb]{0.73,0.40,0.13}{##1}}}

\def\PYZbs{\char`\\}
\def\PYZus{\char`\_}
\def\PYZob{\char`\{}
\def\PYZcb{\char`\}}
\def\PYZca{\char`\^}
\def\PYZam{\char`\&}
\def\PYZlt{\char`\<}
\def\PYZgt{\char`\>}
\def\PYZsh{\char`\#}
\def\PYZpc{\char`\%}
\def\PYZdl{\char`\$}
\def\PYZhy{\char`\-}
\def\PYZsq{\char`\'}
\def\PYZdq{\char`\"}
\def\PYZti{\char`\~}
% for compatibility with earlier versions
\def\PYZat{@}
\def\PYZlb{[}
\def\PYZrb{]}
\makeatother


    % Exact colors from NB
    \definecolor{incolor}{rgb}{0.0, 0.0, 0.5}
    \definecolor{outcolor}{rgb}{0.545, 0.0, 0.0}



    
    % Prevent overflowing lines due to hard-to-break entities
    \sloppy 
    % Setup hyperref package
    \hypersetup{
      breaklinks=true,  % so long urls are correctly broken across lines
      colorlinks=true,
      urlcolor=blue,
      linkcolor=darkorange,
      citecolor=darkgreen,
      }
    % Slightly bigger margins than the latex defaults
    
    \geometry{verbose,tmargin=1in,bmargin=1in,lmargin=1in,rmargin=1in}
    
    

    \begin{document}
    
    
    \maketitle
    
    

    

    \section{QuTiP example: Physical implementation of Spin Chain Qubit model}


    Author: Anubhav Vardhan (anubhavvardhan@gmail.com)

For more information about QuTiP see \url{http://qutip.org}

    \begin{Verbatim}[commandchars=\\\{\}]
{\color{incolor}In [{\color{incolor}1}]:} \PY{o}{\PYZpc{}}\PY{k}{pylab} \PY{n}{inline}
\end{Verbatim}

    \begin{Verbatim}[commandchars=\\\{\}]
Populating the interactive namespace from numpy and matplotlib
    \end{Verbatim}

    \begin{Verbatim}[commandchars=\\\{\}]
{\color{incolor}In [{\color{incolor}2}]:} \PY{k+kn}{from} \PY{n+nn}{qutip} \PY{k+kn}{import} \PY{o}{*}
\end{Verbatim}

    \begin{Verbatim}[commandchars=\\\{\}]
{\color{incolor}In [{\color{incolor}3}]:} \PY{k+kn}{from} \PY{n+nn}{qutip.qip.models.circuitprocessor} \PY{k+kn}{import} \PY{o}{*}
\end{Verbatim}

    \begin{Verbatim}[commandchars=\\\{\}]
{\color{incolor}In [{\color{incolor}4}]:} \PY{k+kn}{from} \PY{n+nn}{qutip.qip.models.spinchain} \PY{k+kn}{import} \PY{o}{*}
\end{Verbatim}


    \subsection{Hamiltonian:}


    $\displaystyle H = - \frac{1}{2}\sum_n^N h_n \sigma_z(n) - \frac{1}{2} \sum_n^{N-1} [ J_x^{(n)} \sigma_x(n) \sigma_x(n+1) + J_y^{(n)} \sigma_y(n) \sigma_y(n+1) +J_z^{(n)} \sigma_z(n) \sigma_z(n+1)]$

    The linear and circular spin chain models employing the nearest neighbor
interaction can be implemented using the SpinChain class.


    \subsection{Circuit Setup}


    \begin{Verbatim}[commandchars=\\\{\}]
{\color{incolor}In [{\color{incolor}5}]:} \PY{n}{N} \PY{o}{=} \PY{l+m+mi}{6}
        \PY{n}{qc} \PY{o}{=} \PY{n}{QubitCircuit}\PY{p}{(}\PY{n}{N}\PY{p}{)}
        
        \PY{n}{qc}\PY{o}{.}\PY{n}{add\PYZus{}gate}\PY{p}{(}\PY{l+s}{\PYZdq{}}\PY{l+s}{CNOT}\PY{l+s}{\PYZdq{}}\PY{p}{,} \PY{n}{targets}\PY{o}{=}\PY{p}{[}\PY{l+m+mi}{1}\PY{p}{]}\PY{p}{,} \PY{n}{controls}\PY{o}{=}\PY{p}{[}\PY{l+m+mi}{5}\PY{p}{]}\PY{p}{)}
        
        \PY{n}{qc}\PY{o}{.}\PY{n}{png}
\end{Verbatim}
\texttt{\color{outcolor}Out[{\color{outcolor}5}]:}
    
    \begin{center}
    \adjustimage{max size={0.9\linewidth}{0.9\paperheight}}{example-spin-chain-model_files/example-spin-chain-model_10_0.png}
    \end{center}
    { \hspace*{\fill} \\}
    

    The non-adjacent interactions are broken into a series of adjacent ones
by the program automatically.

    \begin{Verbatim}[commandchars=\\\{\}]
{\color{incolor}In [{\color{incolor}6}]:} \PY{n}{U\PYZus{}ideal} \PY{o}{=} \PY{n}{gate\PYZus{}sequence\PYZus{}product}\PY{p}{(}\PY{n}{qc}\PY{o}{.}\PY{n}{unitary\PYZus{}matrix}\PY{p}{(}\PY{p}{)}\PY{p}{)}
        
        \PY{n}{U\PYZus{}ideal}
\end{Verbatim}
\texttt{\color{outcolor}Out[{\color{outcolor}6}]:}
    
    Quantum object: dims = [[2, 2, 2, 2, 2, 2], [2, 2, 2, 2, 2, 2]], shape = [64, 64], type = oper, isherm = True\begin{equation*}\begin{pmatrix}1.0 & 0.0 & 0.0 & 0.0 & 0.0 & \cdots & 0.0 & 0.0 & 0.0 & 0.0 & 0.0\\0.0 & 0.0 & 0.0 & 0.0 & 0.0 & \cdots & 0.0 & 0.0 & 0.0 & 0.0 & 0.0\\0.0 & 0.0 & 1.0 & 0.0 & 0.0 & \cdots & 0.0 & 0.0 & 0.0 & 0.0 & 0.0\\0.0 & 0.0 & 0.0 & 0.0 & 0.0 & \cdots & 0.0 & 0.0 & 0.0 & 0.0 & 0.0\\0.0 & 0.0 & 0.0 & 0.0 & 1.0 & \cdots & 0.0 & 0.0 & 0.0 & 0.0 & 0.0\\\vdots & \vdots & \vdots & \vdots & \vdots & \ddots & \vdots & \vdots & \vdots & \vdots & \vdots\\0.0 & 0.0 & 0.0 & 0.0 & 0.0 & \cdots & 0.0 & 0.0 & 0.0 & 0.0 & 0.0\\0.0 & 0.0 & 0.0 & 0.0 & 0.0 & \cdots & 0.0 & 1.0 & 0.0 & 0.0 & 0.0\\0.0 & 0.0 & 0.0 & 0.0 & 0.0 & \cdots & 0.0 & 0.0 & 0.0 & 0.0 & 0.0\\0.0 & 0.0 & 0.0 & 0.0 & 0.0 & \cdots & 0.0 & 0.0 & 0.0 & 1.0 & 0.0\\0.0 & 0.0 & 0.0 & 0.0 & 0.0 & \cdots & 0.0 & 0.0 & 0.0 & 0.0 & 0.0\\\end{pmatrix}\end{equation*}

    


    \subsection{Circular Spin Chain Model Implementation}


    \begin{Verbatim}[commandchars=\\\{\}]
{\color{incolor}In [{\color{incolor}7}]:} \PY{n}{p1} \PY{o}{=} \PY{n}{CircularSpinChain}\PY{p}{(}\PY{n}{N}\PY{p}{,} \PY{n}{correct\PYZus{}global\PYZus{}phase}\PY{o}{=}\PY{n+nb+bp}{True}\PY{p}{)}
        
        \PY{n}{U\PYZus{}list} \PY{o}{=} \PY{n}{p1}\PY{o}{.}\PY{n}{run}\PY{p}{(}\PY{n}{qc}\PY{p}{)}
        
        \PY{n}{U\PYZus{}physical} \PY{o}{=} \PY{n}{gate\PYZus{}sequence\PYZus{}product}\PY{p}{(}\PY{n}{U\PYZus{}list}\PY{p}{)}
        
        \PY{n}{U\PYZus{}physical}\PY{o}{.}\PY{n}{tidyup}\PY{p}{(}\PY{n}{atol}\PY{o}{=}\PY{l+m+mf}{1e\PYZhy{}5}\PY{p}{)}
\end{Verbatim}
\texttt{\color{outcolor}Out[{\color{outcolor}7}]:}
    
    Quantum object: dims = [[2, 2, 2, 2, 2, 2], [2, 2, 2, 2, 2, 2]], shape = [64, 64], type = oper, isherm = True\begin{equation*}\begin{pmatrix}1.000 & 0.0 & 0.0 & 0.0 & 0.0 & \cdots & 0.0 & 0.0 & 0.0 & 0.0 & 0.0\\0.0 & 0.0 & 0.0 & 0.0 & 0.0 & \cdots & 0.0 & 0.0 & 0.0 & 0.0 & 0.0\\0.0 & 0.0 & 1.000 & 0.0 & 0.0 & \cdots & 0.0 & 0.0 & 0.0 & 0.0 & 0.0\\0.0 & 0.0 & 0.0 & 0.0 & 0.0 & \cdots & 0.0 & 0.0 & 0.0 & 0.0 & 0.0\\0.0 & 0.0 & 0.0 & 0.0 & 1.000 & \cdots & 0.0 & 0.0 & 0.0 & 0.0 & 0.0\\\vdots & \vdots & \vdots & \vdots & \vdots & \ddots & \vdots & \vdots & \vdots & \vdots & \vdots\\0.0 & 0.0 & 0.0 & 0.0 & 0.0 & \cdots & 0.0 & 0.0 & 0.0 & 0.0 & 0.0\\0.0 & 0.0 & 0.0 & 0.0 & 0.0 & \cdots & 0.0 & 1.000 & 0.0 & 0.0 & 0.0\\0.0 & 0.0 & 0.0 & 0.0 & 0.0 & \cdots & 0.0 & 0.0 & 0.0 & 0.0 & 0.0\\0.0 & 0.0 & 0.0 & 0.0 & 0.0 & \cdots & 0.0 & 0.0 & 0.0 & 1.000 & 0.0\\0.0 & 0.0 & 0.0 & 0.0 & 0.0 & \cdots & 0.0 & 0.0 & 0.0 & 0.0 & 0.0\\\end{pmatrix}\end{equation*}

    

    \begin{Verbatim}[commandchars=\\\{\}]
{\color{incolor}In [{\color{incolor}8}]:} \PY{p}{(}\PY{n}{U\PYZus{}ideal} \PY{o}{\PYZhy{}} \PY{n}{U\PYZus{}physical}\PY{p}{)}\PY{o}{.}\PY{n}{norm}\PY{p}{(}\PY{p}{)}
\end{Verbatim}

            \begin{Verbatim}[commandchars=\\\{\}]
{\color{outcolor}Out[{\color{outcolor}8}]:} 0.0
\end{Verbatim}
        
    The results obtained from the physical implementation agree with the
ideal result.

    \begin{Verbatim}[commandchars=\\\{\}]
{\color{incolor}In [{\color{incolor}9}]:} \PY{n}{p1}\PY{o}{.}\PY{n}{qc0}\PY{o}{.}\PY{n}{gates}
\end{Verbatim}

            \begin{Verbatim}[commandchars=\\\{\}]
{\color{outcolor}Out[{\color{outcolor}9}]:} [Gate(CNOT, targets=[1], controls=[5])]
\end{Verbatim}
        
    The gates are first convert to gates with adjacent interactions moving
in the direction with the least number of qubits in between.

    \begin{Verbatim}[commandchars=\\\{\}]
{\color{incolor}In [{\color{incolor}10}]:} \PY{n}{p1}\PY{o}{.}\PY{n}{qc1}\PY{o}{.}\PY{n}{gates}
\end{Verbatim}

            \begin{Verbatim}[commandchars=\\\{\}]
{\color{outcolor}Out[{\color{outcolor}10}]:} [Gate(SWAP, targets=[5, 0], controls=None),
          Gate(CNOT, targets=[1], controls=[0]),
          Gate(SWAP, targets=[5, 0], controls=None)]
\end{Verbatim}
        
    They are then converted into the basis {[}ISWAP, RX, RZ{]}

    \begin{Verbatim}[commandchars=\\\{\}]
{\color{incolor}In [{\color{incolor}11}]:} \PY{n}{p1}\PY{o}{.}\PY{n}{qc2}\PY{o}{.}\PY{n}{gates}
\end{Verbatim}

            \begin{Verbatim}[commandchars=\\\{\}]
{\color{outcolor}Out[{\color{outcolor}11}]:} [Gate(GLOBALPHASE, targets=None, controls=None),
          Gate(ISWAP, targets=[5, 0], controls=None),
          Gate(RX, targets=[5], controls=None),
          Gate(ISWAP, targets=[5, 0], controls=None),
          Gate(RX, targets=[0], controls=None),
          Gate(ISWAP, targets=[0, 5], controls=None),
          Gate(RX, targets=[5], controls=None),
          Gate(GLOBALPHASE, targets=None, controls=None),
          Gate(ISWAP, targets=[0, 1], controls=None),
          Gate(RZ, targets=[1], controls=None),
          Gate(RZ, targets=[0], controls=None),
          Gate(RX, targets=[0], controls=None),
          Gate(RZ, targets=[0], controls=None),
          Gate(RZ, targets=[0], controls=None),
          Gate(ISWAP, targets=[0, 1], controls=None),
          Gate(RZ, targets=[1], controls=None),
          Gate(RX, targets=[1], controls=None),
          Gate(RZ, targets=[1], controls=None),
          Gate(RZ, targets=[1], controls=None),
          Gate(GLOBALPHASE, targets=None, controls=None),
          Gate(ISWAP, targets=[5, 0], controls=None),
          Gate(RX, targets=[5], controls=None),
          Gate(ISWAP, targets=[5, 0], controls=None),
          Gate(RX, targets=[0], controls=None),
          Gate(ISWAP, targets=[0, 5], controls=None),
          Gate(RX, targets=[5], controls=None)]
\end{Verbatim}
        
    The time for each applied gate:

    \begin{Verbatim}[commandchars=\\\{\}]
{\color{incolor}In [{\color{incolor}12}]:} \PY{n}{p1}\PY{o}{.}\PY{n}{T\PYZus{}list}
\end{Verbatim}

            \begin{Verbatim}[commandchars=\\\{\}]
{\color{outcolor}Out[{\color{outcolor}12}]:} [1.25,
          0.5,
          1.25,
          0.5,
          1.25,
          0.5,
          1.25,
          0.125,
          0.125,
          0.5,
          0.125,
          0.125,
          1.25,
          0.125,
          0.5,
          0.125,
          0.125,
          1.25,
          0.5,
          1.25,
          0.5,
          1.25,
          0.5]
\end{Verbatim}
        
    The pulse can be plotted as:

    \begin{Verbatim}[commandchars=\\\{\}]
{\color{incolor}In [{\color{incolor}13}]:} \PY{n}{p1}\PY{o}{.}\PY{n}{plot\PYZus{}pulses}\PY{p}{(}\PY{p}{)}
\end{Verbatim}

            \begin{Verbatim}[commandchars=\\\{\}]
{\color{outcolor}Out[{\color{outcolor}13}]:} (<matplotlib.figure.Figure at 0x7fbac956c278>,
          <matplotlib.axes.AxesSubplot at 0x7fbac955e4e0>)
\end{Verbatim}
        
    \begin{center}
    \adjustimage{max size={0.9\linewidth}{0.9\paperheight}}{example-spin-chain-model_files/example-spin-chain-model_25_1.png}
    \end{center}
    { \hspace*{\fill} \\}
    

    \subsection{Linear Spin Chain Model Implementation}


    \begin{Verbatim}[commandchars=\\\{\}]
{\color{incolor}In [{\color{incolor}14}]:} \PY{n}{p2} \PY{o}{=} \PY{n}{LinearSpinChain}\PY{p}{(}\PY{n}{N}\PY{p}{,} \PY{n}{correct\PYZus{}global\PYZus{}phase}\PY{o}{=}\PY{n+nb+bp}{True}\PY{p}{)}
         
         \PY{n}{U\PYZus{}list} \PY{o}{=} \PY{n}{p2}\PY{o}{.}\PY{n}{run}\PY{p}{(}\PY{n}{qc}\PY{p}{)}
         
         \PY{n}{U\PYZus{}physical} \PY{o}{=} \PY{n}{gate\PYZus{}sequence\PYZus{}product}\PY{p}{(}\PY{n}{U\PYZus{}list}\PY{p}{)}
         
         \PY{n}{U\PYZus{}physical}\PY{o}{.}\PY{n}{tidyup}\PY{p}{(}\PY{n}{atol}\PY{o}{=}\PY{l+m+mf}{1e\PYZhy{}5}\PY{p}{)}
\end{Verbatim}
\texttt{\color{outcolor}Out[{\color{outcolor}14}]:}
    
    Quantum object: dims = [[2, 2, 2, 2, 2, 2], [2, 2, 2, 2, 2, 2]], shape = [64, 64], type = oper, isherm = True\begin{equation*}\begin{pmatrix}1.000 & 0.0 & 0.0 & 0.0 & 0.0 & \cdots & 0.0 & 0.0 & 0.0 & 0.0 & 0.0\\0.0 & 0.0 & 0.0 & 0.0 & 0.0 & \cdots & 0.0 & 0.0 & 0.0 & 0.0 & 0.0\\0.0 & 0.0 & 1.000 & 0.0 & 0.0 & \cdots & 0.0 & 0.0 & 0.0 & 0.0 & 0.0\\0.0 & 0.0 & 0.0 & 0.0 & 0.0 & \cdots & 0.0 & 0.0 & 0.0 & 0.0 & 0.0\\0.0 & 0.0 & 0.0 & 0.0 & 1.000 & \cdots & 0.0 & 0.0 & 0.0 & 0.0 & 0.0\\\vdots & \vdots & \vdots & \vdots & \vdots & \ddots & \vdots & \vdots & \vdots & \vdots & \vdots\\0.0 & 0.0 & 0.0 & 0.0 & 0.0 & \cdots & 0.0 & 0.0 & 0.0 & 0.0 & 0.0\\0.0 & 0.0 & 0.0 & 0.0 & 0.0 & \cdots & 0.0 & 1.000 & 0.0 & 0.0 & 0.0\\0.0 & 0.0 & 0.0 & 0.0 & 0.0 & \cdots & 0.0 & 0.0 & 0.0 & 0.0 & 0.0\\0.0 & 0.0 & 0.0 & 0.0 & 0.0 & \cdots & 0.0 & 0.0 & 0.0 & 1.000 & 0.0\\0.0 & 0.0 & 0.0 & 0.0 & 0.0 & \cdots & 0.0 & 0.0 & 0.0 & 0.0 & 0.0\\\end{pmatrix}\end{equation*}

    

    \begin{Verbatim}[commandchars=\\\{\}]
{\color{incolor}In [{\color{incolor}15}]:} \PY{p}{(}\PY{n}{U\PYZus{}ideal} \PY{o}{\PYZhy{}} \PY{n}{U\PYZus{}physical}\PY{p}{)}\PY{o}{.}\PY{n}{norm}\PY{p}{(}\PY{p}{)}
\end{Verbatim}

            \begin{Verbatim}[commandchars=\\\{\}]
{\color{outcolor}Out[{\color{outcolor}15}]:} 0.0
\end{Verbatim}
        
    The results obtained from the physical implementation agree with the
ideal result.

    \begin{Verbatim}[commandchars=\\\{\}]
{\color{incolor}In [{\color{incolor}16}]:} \PY{n}{p2}\PY{o}{.}\PY{n}{qc0}\PY{o}{.}\PY{n}{gates}
\end{Verbatim}

            \begin{Verbatim}[commandchars=\\\{\}]
{\color{outcolor}Out[{\color{outcolor}16}]:} [Gate(CNOT, targets=[1], controls=[5])]
\end{Verbatim}
        
    The gates are first convert to gates with adjacent interactions moving
in the direction with the least number of qubits in between.

    \begin{Verbatim}[commandchars=\\\{\}]
{\color{incolor}In [{\color{incolor}17}]:} \PY{n}{p2}\PY{o}{.}\PY{n}{qc1}\PY{o}{.}\PY{n}{gates}
\end{Verbatim}

            \begin{Verbatim}[commandchars=\\\{\}]
{\color{outcolor}Out[{\color{outcolor}17}]:} [Gate(SWAP, targets=[1, 2], controls=None),
          Gate(SWAP, targets=[4, 5], controls=None),
          Gate(SWAP, targets=[2, 3], controls=None),
          Gate(CNOT, targets=[3], controls=[4]),
          Gate(SWAP, targets=[2, 3], controls=None),
          Gate(SWAP, targets=[4, 5], controls=None),
          Gate(SWAP, targets=[1, 2], controls=None)]
\end{Verbatim}
        
    They are then converted into the basis {[}ISWAP, RX, RZ{]}

    \begin{Verbatim}[commandchars=\\\{\}]
{\color{incolor}In [{\color{incolor}18}]:} \PY{n}{p2}\PY{o}{.}\PY{n}{qc2}\PY{o}{.}\PY{n}{gates}
\end{Verbatim}

            \begin{Verbatim}[commandchars=\\\{\}]
{\color{outcolor}Out[{\color{outcolor}18}]:} [Gate(GLOBALPHASE, targets=None, controls=None),
          Gate(ISWAP, targets=[1, 2], controls=None),
          Gate(RX, targets=[1], controls=None),
          Gate(ISWAP, targets=[1, 2], controls=None),
          Gate(RX, targets=[2], controls=None),
          Gate(ISWAP, targets=[2, 1], controls=None),
          Gate(RX, targets=[1], controls=None),
          Gate(GLOBALPHASE, targets=None, controls=None),
          Gate(ISWAP, targets=[4, 5], controls=None),
          Gate(RX, targets=[4], controls=None),
          Gate(ISWAP, targets=[4, 5], controls=None),
          Gate(RX, targets=[5], controls=None),
          Gate(ISWAP, targets=[5, 4], controls=None),
          Gate(RX, targets=[4], controls=None),
          Gate(GLOBALPHASE, targets=None, controls=None),
          Gate(ISWAP, targets=[2, 3], controls=None),
          Gate(RX, targets=[2], controls=None),
          Gate(ISWAP, targets=[2, 3], controls=None),
          Gate(RX, targets=[3], controls=None),
          Gate(ISWAP, targets=[3, 2], controls=None),
          Gate(RX, targets=[2], controls=None),
          Gate(GLOBALPHASE, targets=None, controls=None),
          Gate(ISWAP, targets=[4, 3], controls=None),
          Gate(RZ, targets=[3], controls=None),
          Gate(RZ, targets=[4], controls=None),
          Gate(RX, targets=[4], controls=None),
          Gate(RZ, targets=[4], controls=None),
          Gate(RZ, targets=[4], controls=None),
          Gate(ISWAP, targets=[4, 3], controls=None),
          Gate(RZ, targets=[3], controls=None),
          Gate(RX, targets=[3], controls=None),
          Gate(RZ, targets=[3], controls=None),
          Gate(RZ, targets=[3], controls=None),
          Gate(GLOBALPHASE, targets=None, controls=None),
          Gate(ISWAP, targets=[2, 3], controls=None),
          Gate(RX, targets=[2], controls=None),
          Gate(ISWAP, targets=[2, 3], controls=None),
          Gate(RX, targets=[3], controls=None),
          Gate(ISWAP, targets=[3, 2], controls=None),
          Gate(RX, targets=[2], controls=None),
          Gate(GLOBALPHASE, targets=None, controls=None),
          Gate(ISWAP, targets=[4, 5], controls=None),
          Gate(RX, targets=[4], controls=None),
          Gate(ISWAP, targets=[4, 5], controls=None),
          Gate(RX, targets=[5], controls=None),
          Gate(ISWAP, targets=[5, 4], controls=None),
          Gate(RX, targets=[4], controls=None),
          Gate(GLOBALPHASE, targets=None, controls=None),
          Gate(ISWAP, targets=[1, 2], controls=None),
          Gate(RX, targets=[1], controls=None),
          Gate(ISWAP, targets=[1, 2], controls=None),
          Gate(RX, targets=[2], controls=None),
          Gate(ISWAP, targets=[2, 1], controls=None),
          Gate(RX, targets=[1], controls=None)]
\end{Verbatim}
        
    The time for each applied gate:

    \begin{Verbatim}[commandchars=\\\{\}]
{\color{incolor}In [{\color{incolor}19}]:} \PY{n}{p2}\PY{o}{.}\PY{n}{T\PYZus{}list}
\end{Verbatim}

            \begin{Verbatim}[commandchars=\\\{\}]
{\color{outcolor}Out[{\color{outcolor}19}]:} [1.25,
          0.5,
          1.25,
          0.5,
          1.25,
          0.5,
          1.25,
          0.5,
          1.25,
          0.5,
          1.25,
          0.5,
          1.25,
          0.5,
          1.25,
          0.5,
          1.25,
          0.5,
          1.25,
          0.125,
          0.125,
          0.5,
          0.125,
          0.125,
          1.25,
          0.125,
          0.5,
          0.125,
          0.125,
          1.25,
          0.5,
          1.25,
          0.5,
          1.25,
          0.5,
          1.25,
          0.5,
          1.25,
          0.5,
          1.25,
          0.5,
          1.25,
          0.5,
          1.25,
          0.5,
          1.25,
          0.5]
\end{Verbatim}
        
    The pulse can be plotted as:

    \begin{Verbatim}[commandchars=\\\{\}]
{\color{incolor}In [{\color{incolor}20}]:} \PY{n}{p2}\PY{o}{.}\PY{n}{plot\PYZus{}pulses}\PY{p}{(}\PY{p}{)}
\end{Verbatim}

    \begin{Verbatim}[commandchars=\\\{\}]

        ---------------------------------------------------------------------------
    IndexError                                Traceback (most recent call last)

        <ipython-input-20-7aab6ae688ac> in <module>()
    ----> 1 p2.plot\_pulses()
    

        /usr/local/lib/python3.4/dist-packages/qutip/qip/models/circuitprocessor.py in plot\_pulses(self)
        213 
        214         for n, uu in enumerate(u):
    --> 215             ax.plot(t, u[n], label=u\_labels[n])
        216 
        217         ax.axis('tight')


        IndexError: list index out of range

    \end{Verbatim}

    \begin{center}
    \adjustimage{max size={0.9\linewidth}{0.9\paperheight}}{example-spin-chain-model_files/example-spin-chain-model_38_1.png}
    \end{center}
    { \hspace*{\fill} \\}
    

    \subsubsection{Software versions:}


    \begin{Verbatim}[commandchars=\\\{\}]
{\color{incolor}In [{\color{incolor}21}]:} \PY{k+kn}{from} \PY{n+nn}{qutip.ipynbtools} \PY{k+kn}{import} \PY{n}{version\PYZus{}table}
         \PY{n}{version\PYZus{}table}\PY{p}{(}\PY{p}{)}
\end{Verbatim}

            \begin{Verbatim}[commandchars=\\\{\}]
{\color{outcolor}Out[{\color{outcolor}21}]:} <IPython.core.display.HTML at 0x7fbac8e802b0>
\end{Verbatim}
        

    % Add a bibliography block to the postdoc
    
    
    
    \end{document}
